\documentclass[11pt,]{article}
\usepackage[margin=1in]{geometry}
\newcommand*{\authorfont}{\fontfamily{phv}\selectfont}
\usepackage[]{fourier}
\usepackage{multicol}
\usepackage{abstract}
\renewcommand{\abstractname}{}    % clear the title
\renewcommand{\absnamepos}{empty} % originally center
\newcommand{\blankline}{\quad\pagebreak[2]}

\usepackage[dvipsnames]{xcolor}
\usepackage[margin=1in]{geometry}
\usepackage{graphicx}

\definecolor{unhtext}{RGB}{1, 71, 133}


\providecommand{\tightlist}{%
  \setlength{\itemsep}{0pt}\setlength{\parskip}{0pt}} 
\usepackage{longtable,booktabs}
      
\usepackage{parskip}
\usepackage[pagestyles,raggedright]{titlesec}
\usepackage{titletoc}
\usepackage{blindtext}
\titlespacing\section{0pt}{12pt plus 4pt minus 2pt}{6pt plus 2pt minus 2pt}
\titlespacing\subsection{0pt}{12pt plus 4pt minus 2pt}{6pt plus 2pt minus 2pt}

\titleformat*{\subsubsection}{\normalsize\itshape\color{unhtext}}
\titleformat*{\section}{\Large\bfseries\color{unhtext}}
\titleformat*{\subsection}{\large\bfseries\color{unhtext}}


\usepackage{titling}
\pretitle{\begin{center} \includegraphics[height=2cm]{images/logo5.png}\LARGE\\}
\posttitle{\end{center}}
\setlength{\droptitle}{-.15cm}

%\setlength{\parindent}{0pt}
%\setlength{\parskip}{6pt plus 2pt minus 1pt}
%\setlength{\emergencystretch}{3em}  % prevent overfull lines 

\usepackage[T1]{fontenc}
\usepackage[utf8]{inputenc}

\usepackage{fancyhdr}
\pagestyle{fancy}
\usepackage{lastpage}
\renewcommand{\headrulewidth}{0.3pt}
\renewcommand{\footrulewidth}{0.0pt} 
\lhead{}
\chead{}
\rhead{\footnotesize \textcolor{unhtext}{Pompeo College of Business --
BANL 6100 -- Fall 2023}}
\lfoot{}
\cfoot{\small \thepage /\pageref*{LastPage}}
\rfoot{}

\fancypagestyle{firststyle}
{
\renewcommand{\headrulewidth}{0pt}%
   \fancyhf{}
   \fancyfoot[C]{\small \thepage/\pageref*{LastPage}}
}

%\def\labelitemi{--}
%\usepackage{enumitem}
%\setitemize[0]{leftmargin=25pt}
%\setenumerate[0]{leftmargin=25pt}


\makeatletter
\@ifpackageloaded{hyperref}{}{%
\ifxetex
  \usepackage[setpagesize=false, % page size defined by xetex
              unicode=false, % unicode breaks when used with xetex
              xetex]{hyperref}
\else
  \usepackage[unicode=true]{hyperref}
\fi
}
\@ifpackageloaded{color}{
    \PassOptionsToPackage{usenames,dvipsnames}{color}
}{%
    \usepackage[usenames,dvipsnames]{color}
}
\makeatother
\hypersetup{breaklinks=true,
            bookmarks=true,
            pdfauthor={},
             pdfkeywords = {},  
            pdftitle={Pompeo College of Business\\
Department of Economics \& Business Analytics: BANL 6100: Business
Analytics},
            colorlinks=true,
            citecolor=blue,
            urlcolor=blue,
            linkcolor=magenta,
            pdfborder={0 0 0}}
\urlstyle{same}  % don't use monospace font for urls


\setcounter{secnumdepth}{0}

\usepackage{longtable}




\usepackage{setspace}

\title{\textcolor{unhtext}{Pompeo College of Business\\
Department of Economics \& Business Analytics}}
\usepackage{etoolbox}
%\providecommand{\subtitle}[1]{}
\makeatletter
\providecommand{\subtitle}[1]{% add subtitle to \maketitle
  \apptocmd{\@title}{\par {\LARGE #1}}{}{}
}
\makeatother
\subtitle{\textcolor{unhtext}{BANL 6100: Business Analytics}}

\date{\textcolor{unhtext}{Fall 2023}}


\begin{document}  

		\maketitle
		
	
		\thispagestyle{firststyle}

%	\thispagestyle{empty}

\vspace{-12pt}
\noindent\Large{\textbf{\textcolor{unhtext}{Course information}}}\vspace{3pt}\\
  Course: BANL 6100-08 -- Business Analytics\\
  Credid hours: 3 credits\\
  Semester: Fall 2023\\
  Classroom: CRN 91887\\
  Class Hours: 11:15 am - 12:05 pm MW

\vspace{10pt}
\noindent\textbf{\Large{\textcolor{unhtext}{Faculty Contact Information}}}
\vspace{-8pt}
\begin{multicols}{2}
  \noindent Dr.~Mehmet Balcilar\\
  E: \href{mailto:MBalcilar@newhaven.edu}{\nolinkurl{MBalcilar@newhaven.edu}}\\
    T: (203) 479-4779
  \\
   
  W: \href{http://www.mbalcilar.net}{\tt www.mbalcilar.net}
  \\
    Office: Orange Campus N125
  \\
    Office Hours: 1:00-3:00 p.m. MTuTh
  \\
    \columnbreak
    \end{multicols}
 
 \vspace{-20pt}

\noindent\textbf{\Large{\textcolor{unhtext}{Department Chair Contact Information}}}\\
\vspace{-26pt}
 \begin{multicols}{2}
\noindent{Dr.~Gazi Duman}\\
  E: \href{mailto:GDuman@newhaven.edu}{\nolinkurl{GDuman@newhaven.edu}}\\
  \vfill\columnbreak
  T: (203) 479-4564\\
  Office: Orange Campus N132A\\
\end{multicols}

\vspace{2mm}

\begin{center}
\textbf{\Large{\textcolor{unhtext}{COURSE SYLLABUS}}}
\end{center}
   
 % body from Rmd file goes here 
\hypertarget{course-description}{%
\section{Course Description}\label{course-description}}

This course reviews statistical concepts and methods with emphasis on
data analytics and visualizations. Topics to be covered include
descriptive statistics, plots and graphs for discrete and continuous
data, statistical inference, regression, and selected non-parametrics
including chi-square. In addition, power pivot and other Excel
analytical tools will be covered. Students will obtain a solid
introduction to R as a functional programming language and will be able
to use Excel and R to effectively compute statistical and graphical
procedures.

\hypertarget{course-format}{%
\section{Course Format}\label{course-format}}

The course will be delivered as a fully on-ground course, with every
student meeting in person.

\hypertarget{required-text}{%
\section{Required Text}\label{required-text}}

\emph{OpenIntro Statistics}, Christopher D. Barr, David M. Diez, and
Mine Çetinkaya-Rundel, 4th Edition, 2019. This is a free eBook,
available at \url{http://openintro.org/os/pdf}.

New and used hard copies of the book can be purchased at
\href{https://www.amazon.com/OpenIntro-Statistics-Fourth-color-interior/dp/1943450226/ref=sr_1_1?crid=XKV3CDPZPZDG\&keywords=OpenIntro+Statistics\&qid=1692970332\&sprefix=openintro+statistics\%2Caps\%2C84\&sr=8-1}{Amazon}.

\hypertarget{required-software}{%
\section{Required Software}\label{required-software}}

All students are rewuired to use
\href{https://www.r-project.org}{\textbf{R}} and
\href{https://posit.co/products/open-source/rstudio/}{\textbf{RStudio}},
a no-cost add-on coding environment for R provided by
\href{https://posit.co}{\textbf{Posit}}. R serves as a statistical
programming language, while R Studio operates as a third-party software
user interface designed for R. These tools will find application in
nearly all course assignments. R is available under the GNP General
Public License, compatible with various operating systems, including
Windows and Mac OS. Similarly, RStudio can be downloaded free of charge.
The computer lab has already installed both R and R-Studio. You have the
option to download them onto your personal computer as outlined below:

\begin{itemize}
\item[] \textbf{R:} \href{https://cran.r-project.org}{https://cran.r-project.org}
\item[] \textbf{R-Studio:} \href{https://posit.co/download/rstudio-desktop/}{https://posit.co/download/rstudio-desktop}
\end{itemize}

Posit web page for RStudio will suggest the appropriate version for your
operating system. Furthermore, the College of Business has acquired a
license to locally operate the RStudio server at the University of New
Haven. This server is accessible at

\begin{center}
{\bfseries \href{rstudio.newhaven.edu:8787}{rstudio.newhaven.edu:8787} }
\end{center}

Additionally, it serves as a safeguard, ensuring that all R scripts are
preserved in the event that any files are inadvertently lost within your
desktop directories. You will be provide with your ID and password
during the first week of the class. Using the online server presents a
viable alternative in case you encounter technical difficulties with R
on your personal computer. The RStdio Server offers enhanced
convenience, allowing access to R from home, the classroom, or the
library using any browser, computer, or tablet. Moreover, it facilitates
the centralization of all your coursework in one location.

Posit also offers a free cloud bases RStudio server. For more
information, please refer to: \url{https://posit.cloud/plans/free}.

\hypertarget{additional-resources}{%
\section{Additional Resources}\label{additional-resources}}

\begin{itemize}
\tightlist
\item
  You can follow the steps at
  \href{https://posit.co/download/rstudio-desktop/}{Posit} to install R
  and RStuio on your computer.
\item
  \href{https://technology.newhaven.edu/linkedin-learning/}{LinkedIn
  Learning}, provided by the University of New Haven provides University
  faculty, staff, and students with free access to thousands of
  top-quality video tutorials. You can access
  \href{https://technology.newhaven.edu/linkedin-learning/}{New Haven
  LinkedIn}
  \href{https://technology.newhaven.edu/linkedin-learning/}{here}. One
  of the courses you may helpful is
  \href{https://www.linkedin.com/learning/paths/getting-started-with-r-for-data-science?u=2359714}{Getting
  Started with R for Data Science}, which is available
  \href{https://www.linkedin.com/learning/paths/getting-started-with-r-for-data-science?u=2359714}{here}
\item
  Youtube videos, such as MarinStatsLectures channel:
  \url{https://www.youtube.com/channel/UCaNIxVagLhqupvUiDK01Mgg}
\item
  \href{https://www.r-bloggers.com}{R-Bloggers} is web site to keep
  up-to-date with R and analytics: \url{https://www.r-bloggers.com}.
\item
  \href{https://stackoverflow.com/questions}{Stack Overflow} is public
  platform serving 100 million people every month about technical
  questions and solutions suggested by users. An answer to a question
  you have is. very likely to exists here or you can ask your question
  id not already found.
\end{itemize}

\hypertarget{course-goals}{%
\section{Course Goals}\label{course-goals}}

The goals of this course are:

\begin{enumerate}
\def\labelenumi{\arabic{enumi}.}
\tightlist
\item
  to help students understand basic statistics and analytics concepts,
  and
\item
  to help students learn fundamental tools that will allow them to
  tackle with business
\end{enumerate}

analytics challenges.

\hypertarget{course-learning-objectives}{%
\section{Course Learning Objectives}\label{course-learning-objectives}}

At the end of this course, students should be able to:

\begin{enumerate}
\def\labelenumi{\arabic{enumi}.}
\tightlist
\item
  demonstrate basic statistical skills and data exploration,
\item
  conduct data analysis using appropriate statistical software, and
\item
  present data analysis graphically.
\end{enumerate}

\hypertarget{course-requirements-assessment}{%
\section{Course Requirements \&
Assessment}\label{course-requirements-assessment}}

Describe course assignments and assessment and where/when/if a camera
will be required for each assignment and assessment.

\hypertarget{expectations}{%
\subsection{Expectations}\label{expectations}}

This course will require significant in-class and out-of-class
commitment from each student. The University estimates that a student
should expect to spend two hours outside of class for each hour they are
in a class. (For example, a three-credit course would average six
{[}6{]} hours of additional work \underline{outside} of class).

This course will teach you coding in \texttt{R} language at the
intermediate level. Learning coding could be challenging in the
beginning. You are expected to practice coding on a continuous basis and
ask help from the instructor/TAs when you encounter an issue. Learning
statistical concepts and analytics will occur simultaneous to students
learning \texttt{R}. It is advantageous, but not required to know basis
statistical concepts.

\hypertarget{exams}{%
\subsection{Exams}\label{exams}}

There will be a midterm and a final exam. Check the schedule below for
the exam dates. Contact the instructor if you have any schedule conflict
with another course. All exams will be taken during class session.

\hypertarget{homework}{%
\subsection{Homework}\label{homework}}

There will be several homework assignments during the semester. Your
answer for each homework should be uploaded to Canvas by the due date.
If a submission is late, it will receive no credit. Contact the
instructor if you believe that extenuating circumstances prevail but be
aware that exceptions will be made only for truly exceptional
situations. Extra credit homework could be assigned at the discretion of
the instructor.

\hypertarget{datacamp.com}{%
\subsection{Datacamp.com}\label{datacamp.com}}

It is strongly recommended that you finish a free course on statistical
programming language \texttt{R} to speed up your understanding of coding
in \texttt{R}. The name of this online course is
\texttt{Introduction\ to\ R} and provided by \url{datacamp.com}. You
receive a Certificate of Completion at the end. Also, you will get one
full homework credits for completing this course. Register for
\url{datacamp.com} using
\href{https://www.datacamp.com/groups/shared_links/47a7ed52e9916a007f7d4935715bcb752027ec071fca98fb44eb10f407c7324}{this
link}. You may also receive an invitation from the course (arranged by
the instructor). One side benefit signing up for \url{datacamp.com} is
that all other courses offered at this site will be free for you for 6
months.

\hypertarget{extra-credit}{%
\subsection{Extra Credit}\label{extra-credit}}

All students will have periodic opportunities for extra credit
throughout the semester. Exams typically include extra credit inquiries,
for instance. Additionally, there will be extra credit practice
assignments. For reasons of fairness, no individual opportunities for
extra credit will be offered.

\hypertarget{attendance-and-class-participation}{%
\subsection{Attendance and Class
Participation}\label{attendance-and-class-participation}}

You are expected to participate in all class activities, and you should
be prepared to contribute to the discussions and class work. This class
moves at a demanding pace and requires consistent attendance to be
successful. Attendance will be taken but will not be graded. It is
recognized that absences may be unavoidable due to illness or family
emergency. In this event, students are expected to keep up with their
readings and assignments. Absences for more than two (2) weeks is
detrimental to a student's ability to keep up with the course work. In
this event, the instructor's permission is required for satisfactory
completion of the course. Whether excused or not, excessive absences may
result in a failing grade since these signify a failure to fulfill
course requirements. Attendance will be recorded via the Qwickly
Attendance application that is embedded in Canvas. Students will receive
a pin code to record their attendance. If a student fails to log their
attendance in Qwickly, then it is their responsibility to get with the
instructor within 48 hours of the class to correct the attendance
reporting issue.

\hypertarget{course-outlineschedule}{%
\section{Course Outline/Schedule}\label{course-outlineschedule}}

\hypertarget{topics}{%
\subsection{Topics}\label{topics}}

\hypertarget{course-schedule}{%
\subsection{Course Schedule}\label{course-schedule}}

This schedule is informational in nature and subject to change due to
unforeseen circumstances, as a result of any circumstance outside the
University's control, or as other needs arise.

\begin{longtable}[]{@{}lccl@{}}
\caption{Course Schedule.}\tabularnewline
\toprule\noalign{}
Date & Class & Week & Lecture \\
\midrule\noalign{}
\endfirsthead
\toprule\noalign{}
Date & Class & Week & Lecture \\
\midrule\noalign{}
\endhead
\bottomrule\noalign{}
\endlastfoot
2023-08-30 & 1 & 1 & 1 \\
2023-09-06 & 2 & 2 & 2 \\
2023-09-13 & 3 & 3 & 3 \\
2023-09-20 & 4 & 4 & 4 \\
2023-09-27 & 5 & 5 & 5 \\
2023-10-04 & 6 & 6 & 6 \\
2023-10-11 & 7 & 7 & 7 \\
2023-10-18 & 8 & 8 & 8 \\
2023-10-25 & 9 & 9 & 9 \\
2023-11-01 & 10 & 10 & 10 \\
2023-11-08 & 11 & 11 & 11 \\
2023-11-15 & 12 & 12 & 12 \\
2023-11-29 & 13 & 14 & 13 \\
2023-12-06 & 14 & 15 & 14 \\
\end{longtable}

\hypertarget{grading}{%
\section{Grading}\label{grading}}

Grades earned are based on your performance on homework, quizzes,
midterm and the final exam.

\begin{longtable}[]{@{}lr@{}}
\toprule\noalign{}
Letter Grade & Grades Scored Between \\
\midrule\noalign{}
\endhead
\bottomrule\noalign{}
\endlastfoot
A+ & 97 to 100 \\
A & 93 to Less than 97 \\
A- & 90 to Less than 93 \\
B+ & 87 to Less than 90 \\
B & 83 to Less than 87 \\
B- & 80 to Less than 83 \\
C+ & 77 to Less than 80 \\
C & 73 to Less than 77 \\
C- & 70 to Less than 73 \\
F & Less than 70 \\
\end{longtable}

Assignment weighting will follow:

\begin{longtable}[]{@{}lr@{}}
\toprule\noalign{}
Item & Weight \\
\midrule\noalign{}
\endhead
\bottomrule\noalign{}
\endlastfoot
Writing scaffolding & 20 \% \\
Semester Project & 25 \% \\
Labratory Assignments & 20 \% \\
Midterm Exam & 15 \% \\
Final Exam & 20 \% \\
\end{longtable}

\hypertarget{diversity-statement}{%
\section{Diversity Statement}\label{diversity-statement}}

The University of New Haven embraces diversity and recognizes our
responsibility to foster a diverse, inclusive, and welcoming environment
in which all members of the Charger community of all backgrounds and
identities can learn, work, and live together. We benefit from the
academic, social, and cultural developments that arise from a diverse
campus that is committed to equity, inclusion, belonging, and
accountability.

We have a responsibility as a community and as individuals to address
and remove barriers, achieve success, and sustain a culture of
inclusivity, empathy, kindness, and compassion. We encourage, welcome,
and embrace participation in ongoing dialogue, engagement, and education
to critically examine and thoughtfully respond to the changing realities
of our community. Diversity, equity, inclusion, acceptance, and
belonging enrich the Charger community and are instrumental to
institutional success and fulfillment of the University mission.

\hypertarget{reporting-bias-incidents}{%
\section{Reporting Bias Incidents}\label{reporting-bias-incidents}}

At the University of New Haven, there is an expectation that all
community members are committed to creating and supporting a climate
which promotes civility, mutual respect, and open-mindedness. There also
exists an understanding that with the freedom of expression comes the
responsibility to support community members' right to live and work in
an environment free from harassment and fear. It is expected that all
members of the University community will engage in anti-bias behavior
and refrain from actions that intimidate, humiliate, or demean persons
or groups or that undermine their security or self-esteem.

If you have an immediate safety concern for yourself or others, and/or
believe someone poses an immediate threat to themselves or others,
please contact University Police at 203-932-7070 or call 911. Community
members can report bias-motivated incidents by completing the form at
\url{www.newhaven.edu/biasreporting}. Community members are encouraged
to complete this form if they are the target of bias or harassing
behaviors, witness such behaviors, or gain knowledge of these behaviors
occurring within the University community. All matters concerning bias
and harassment will be handled by the Dean of Students Office and Human
Resources Office.

\hypertarget{university-wide-academic-policies}{%
\section{\texorpdfstring{\href{https://unh-web-01.newhaven.edu//mycharger/AcademicPoliciesinSyllabi.pdf}{University-wide
Academic
Policies}}{University-wide Academic Policies}}\label{university-wide-academic-policies}}

A continually-updated list of
\href{https://unh-web-01.newhaven.edu//mycharger/AcademicPoliciesinSyllabi.pdf}{University-wide
Academic Policies} and descriptions of key university student resources,
can be found on Canvas. You can access them by simply clicking on the
\texttt{(?)} help button.

The University-wide academic policies include (but are not limited to)
the University's attendance policy, procedures for both adding /
dropping a course and course withdrawals, an explanation for the sorts
of circumstances where incomplete (INC) grades could be considered by
the faculty, and the academic integrity policy (among others). Also in
this location you will find information regarding the process for
reporting bias and topics related to our maintaining a positive learning
environment (including, but not limited to, discrimination and sexual
misconduct).

The list of key university student resources to enable learning include
(but are not limited to) the University's Center for Student Success,
Writing Center, Center for Learning Resources, and the Accessibility
Resource Center.

\newpage

\begin{center} 
\includegraphics[height=4cm]{images/logo6.jpg}
\end{center}

\begin{center} 
\bfseries{\Large{\textcolor{unhtext}{UNIVERSITY STUDENT SUPPORT SERVICES}}}
\end{center}

The University recognizes that students can often use some help outside
of class and offers academic assistance through several offices.

\hypertarget{accessibility-resources-center}{%
\section{\texorpdfstring{\href{https://mycharger.newhaven.edu/web/mycharger/accessibility-resources-center}{Accessibility
Resources
Center}}{Accessibility Resources Center}}\label{accessibility-resources-center}}

The University of New Haven seeks to maintain a supportive academic
environment for all students inclusive of those with disabilities
including chronic health-related conditions and military
service-connected disorders. If you feel that you may need reasonable
accommodations to enable your full participation in this course, please
provide me with your Verification of Reasonable Accommodations letter
through AIM found in MyCharger or contact the Accessibility Resources
Center to begin the process to ensure that accommodations can be made
available to you. Reasonable accommodations are not required to be
provided retroactively and may not be made without written verification
from the Accessibility Resources Center. The Accessibility Resources
Center is located in Sheffield Hall on the ground floor in the rear of
the building, and can be reached by email at
\href{mailto:ARC@newhaven.edu}{\nolinkurl{ARC@newhaven.edu}} or by phone
at (203) 932-7332.

\hypertarget{center-for-learning-resources-clr}{%
\section{\texorpdfstring{\href{https://mycharger.newhaven.edu/web/mycharger/center-for-learning-resources}{Center
for Learning Resources
(CLR)}}{Center for Learning Resources (CLR)}}\label{center-for-learning-resources-clr}}

The Center for Learning Resources (CLR), located in the Peterson
Library, provides academic content support to the students of the
University of New Haven using metacognitive strategies that help
students become aware of and learn to apply optimal learning processes
in the pursuit of creating independent learners. CLR tutors focus
sessions on discussions of concepts and processes and typically use
external examples to help students grasp and apply the material. We
offer both in-person and online tutoring. To make an appointment, call
us at 203-932-7215, write to us at
\href{mailto:clr@newhaven.edu}{\nolinkurl{clr@newhaven.edu}}, or
\href{https://mycharger.newhaven.edu/documents/10354/0/CLR+Student+Appointment1.pdf/85d9dd6f-6765-4f0c-809b-c0dc47ad3b12}{download
the Navigate app}.

\hypertarget{center-for-student-success-css}{%
\section{\texorpdfstring{\href{https://mycharger.newhaven.edu/web/mycharger/fysc-students?inheritRedirect=true}{Center
for Student Success
(CSS)}}{Center for Student Success (CSS)}}\label{center-for-student-success-css}}

The Center for Student Success can help you refine your study skills and
develop new academic strategies. CSS staff assists with enhancing your
time management and organizational skills. They provide understanding of
your GPA, degree audit, and transcripts, and can answer general
questions about academic policies. They also can connect you to campus
resources and assist you with resolving issues as they arise. During
registration periods, CSS advisors work in conjunction your faculty
advisor to provide assistance with the advising and registration
process. Finally, at various points throughout the semester, CSS works
to provide students with progress reports from their instructors.
Students can make an appointment to see a CSS staff member through
\href{https://mycharger.newhaven.edu/web/mycharger/for-students}{Navigate};
the Center for Student Success can be reached via email at
\href{mailto:css@newhaven.edu}{\nolinkurl{css@newhaven.edu}}.

\hypertarget{counseling-psychological-services-caps}{%
\section{\texorpdfstring{\href{https://mycharger.newhaven.edu/web/mycharger/counseling-and-psychological-services}{Counseling
\& Psychological Services
(CAPS)}}{Counseling \& Psychological Services (CAPS)}}\label{counseling-psychological-services-caps}}

CAPS mission is to support the mental health care of students at the
Univeristy. Our services are included in tuition, confidential, and
include individual and group therapy, support groups, consultations, and
24/7 crisis support. We are available in person at Charger Plaza and
remotely, and are in the office M-F, 8:30-4:30. Please call us to
schedule an appointment or with any questions at 203-932-7333; you can
also schedule
\href{https://titanium-web.newhaven.edu/TitaniumWeb-CC/}{online}. If you
experience a mental health crisis after hours, you can call our main
number for support.

\hypertarget{myatt-center}{%
\section{\texorpdfstring{\href{https://mycharger.newhaven.edu/web/mycharger/center-for-diversity-and-inclusion}{Myatt
Center}}{Myatt Center}}\label{myatt-center}}

The Myatt Center for Diversity and Inclusion is committed to creating a
multicultural environment through intentional education, campus
community engagement, and valuing the unique identities of each member
of the Charger Community. Our commitment to diversity is driven by the
core values of connection, belonging, inclusivity, equity, acceptance,
and accountability. The Myatt Center's focus is to create a respectful
and inclusive environment based our awareness and ability to engage with
others who are different on many levels including ethnicity, race,
sexual orientation, gender, military, religious belief, and life
experiences. Please contact the Myatt Center at
\href{mailto:cdi@newhaven.edu}{\nolinkurl{cdi@newhaven.edu}} for any and
all questions related to our programs and resources.

\hypertarget{marvin-k.-peterson-library}{%
\section{\texorpdfstring{\href{https://mycharger.newhaven.edu/web/mycharger/library}{Marvin
K. Peterson
Library}}{Marvin K. Peterson Library}}\label{marvin-k.-peterson-library}}

The Library provides access to online databases, e-books, e-journals,
electronic U.S. Government Documents, print books, educational games,
and audiovisual materials. A search can be conducted through many of
these resources at once by using the
\href{https://mycharger.newhaven.edu/web/mycharger/library}{search box}
``\href{https://mycharger.newhaven.edu/web/mycharger/library}{Quicksearch}.'\,'
The Library provides three floors with individual quiet study space,
collaborative group study space, study rooms with technology,
whiteboards, Dell desktops, iMacs, scanners, and printers. The entire
library is a wireless zone.

Librarians assist in locating relevant sources of information for
research papers, thesis, honors thesis, and other projects. Librarians
answer general reference questions and help with effectively evaluating
sources of information.
\href{https://mycharger.newhaven.edu/web/mycharger/ask-a-librarian}{Help
is available} through a Chat Service, with in-person or online research
consultations, and by \href{mailto:LibraryHelp@newhaven.edu}{E-Mail}.
Complete the \href{mailto:LibraryHelp@newhaven.edu}{Research
Consultation Form} to arrange a time convenient for you. Appointments
can also be made by using the Navigate app.

\href{http://libguides.newhaven.edu/home}{LibGuides} are created to
assist students with research. They contain an overview of resources
available through the library, as well as tutorials, subject guides, and
course specific guides.

\hypertarget{university-writing-center}{%
\section{\texorpdfstring{\href{https://mycharger.newhaven.edu/web/mycharger/university-writing-center}{University
Writing
Center}}{University Writing Center}}\label{university-writing-center}}

The mission of the Writing Center is to provide high-quality tutoring to
undergraduate and graduate students as they write for a wide range of
purposes and audiences. Tutors are undergraduate and graduate students
who are majoring in a variety of fields across the University. We are
here to work with you at any stage in the writing process; bring in your
assignment, your ideas, and any writing you've done so far. You can make
an appointment in Navigate or visit us in person in the lower level of
the library. We offer appointments in person and via Zoom.

\hypertarget{military-veteran-services}{%
\section{\texorpdfstring{\href{https://www.newhaven.edu/veterans/}{Military
\& Veteran
Services}}{Military \& Veteran Services}}\label{military-veteran-services}}

The Military \& Veteran Affairs team is here to answer any questions
Student Veterans (both current and prospective), active
duty/reserve/national guard members, and military family members have
regarding transitioning to higher education, VA educational benefits,
formal advising, or to listen to issues pertaining to class. The
University of New Haven's Military \& Veterans Affairs team consists of
full-time staff, part time student employees, and VA Work Study students
whose aim is to assist and support the student veteran population both
on and off campus. These individuals have a dedication to the
development, success, and well-being of the student veteran population
on campus which includes veterans, active-duty military, service members
in the reserves or national guard, and dependents using a veterans GI
Bill. The office advises, guides, and supports this student population
and is available to assist at a moment's notice to address the needs and
concerns of this population.

\hypertarget{final-thoughts}{%
\section{Final thoughts}\label{final-thoughts}}

This document is a roadmap for our semester. We learn about the Earth
together and our individual experiences shape how we interpret and value
data. Like all your classes, you will get out what you put into this
course. Asking for help from one another and your instructors is
important, don't be afraid to ask a question about something you don't
know or if you want to check your knowledge about something you think
you know.

\textbf{If this document is updated, a copy will be supplied to you via
Canvas and changes will be announced in class.}

% include standard section files
\end{document}

\makeatletter
\def\@maketitle{%
  \newpage
%  \null
%  \vskip 2em%
%  \begin{center}%
  \let \footnote \thanks
    {\fontsize{18}{20}\selectfont\raggedright  \setlength{\parindent}{0pt} \@title \par}%
}
%\fi
\makeatother